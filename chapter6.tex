\chapter{Co-optimised Energy Arbitrage with Regulation Participation}
\section{ Methodology }
\section{ Analysis }
\subsection{  Impact of Energy Throughput During Regulation Participation }
\subsection{ Impact of  Contingency Revenue }
Participants offering to provide contingency services are enabled in accordance with the “trapezium” supplied in their offers. While participants will not necessarily be supplying these services until a contingency occurs they are paid in accordance with their enablement.
Each FCAS is procured competitively each 5 min though a bidding process integrated with the energy dispatch process, managed and optimized within NEMDE. 130 MW is procured for raise FCAS services while 120 MW is procured for FCAS lower services within a 5-minute dispatch interval. An accumulated time error of greater than ± 1.5 s may require additional regulation
support of an extra 60 MW/s deviation for mainland.
\subsection{ Validation of Existing Asset Performance  }
\begin{figure}[H]
    \centering
    \makebox[\textwidth][c]{    \includegraphics[width=\textwidth]{"Pictures/Chapter5/Hornsdale Actual Monthly Energy & FCAS Revenue - South Australia 2018".pdf}}
    \caption{Caption}
    \label{fig:my_label}
\end{figure}