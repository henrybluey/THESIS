
\chapterimage{GESS-photo.jpg} % Chapter heading image

\chapter{Introduction}
\section{ Utility Battery Energy Storage Overview }

Australia's National Electricity Market (NEM) incorporates around 40,000 km of transmission lines, and supplies about 200 TWh of electricity to approximately 9 million customers each year \parencite{AEMO_NEM}. Currently the NEM is undergoing a rapid transformation in the mix of generation types. A rapid influx in low-cost wind and utility solar photovoltaics (PV) are shifting the merit order of lowest cost generation. Combined with ageing thermal generation, unseen behind the meter PV generation and a rise in demand response, operational challenges pertaining to the stability of the grid are rising. As a reflection of the technical challenges facing the NEM wholesale spot electricity prices have inherently exhibited an increase in volatility, driving the requirement for energy storage and load shifting. Furthermore increases an influx of low inertia generation, combined with misplaced incentives for primary frequency control have produced an increased demand for FCAS Regulation services. 

\url{http://apvi.org.au/solar-research-conference/wp-content/uploads/2018/12/01_DI_Boyle_K_2018.pdf.pdf}

\url{https://www.aemo.com.au/-/media/Files/Electricity/NEM/Initiatives/Emerging-Generation/EGES_Stakeholder_Paper_Final.pdf}

\section{ Thesis Aim and Structure }
Whilst prior work in this field focus' on the base revenue stream of energy-only arbitrage, few have analyzed the value of battery storage systems co-optimising dispatch across multiple services or value streams in accordance with operational strategies.